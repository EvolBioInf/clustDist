\documentclass[a4paper, english]{article}
\usepackage{graphics,eurosym,latexsym}
\usepackage{listings}
\lstset{columns=fixed,basicstyle=\ttfamily,numbers=left,numberstyle=\tiny,stepnumber=5,breaklines=true}
\usepackage{pst-all}
\usepackage{algorithmic,algorithm}
\usepackage{times}
\usepackage{babel}
\usepackage[nodayofweek]{datetime}
\usepackage[round]{natbib}
\bibliographystyle{plainnat}
\oddsidemargin=0cm
\evensidemargin=0cm
\textwidth=16cm
\textheight=23cm
\begin{document}

\title{\texttt{clustDist} \input{version}: Cluster Distances into Phylogenies}
\author{\input{author}}
\input{date}
\date{\displaydate{tagDate}}
\maketitle

\section{Introduction} 
Of the many methods available for phylogeny reconstruction, distance
methods are perhaps the simplest. They take as input a matrix of
pairwise distances and generate as output the corresponding cluster
diagram, or phylogeny~\citep{fel04:inf}. The program
\texttt{clustDist} implements two distance methods, UPGMA and neighbor
joining. It takes as input a distance matrix in PHYLIP format and
returns a tree in Newick format~\citep{fel05:phy}. The following
section explains how to compile and test \texttt{clustDist}. In the
subsequent tutorial the program is briefly demonstrated.


\section{Getting Started}
The program \texttt{clustDist} was written in C on a computer running Linux.
Please contact \texttt{\input{email}} if there are any problems
with the program.
\begin{itemize}
\item Obtain the package\\
\texttt{git clone https://www.github.com/\input{repo}/clustDist}
\item Change into the directory just downloaded
\begin{verbatim}
cd clustDist
\end{verbatim}
and make \texttt{clustDist}
\begin{verbatim}
make
\end{verbatim}
\item Test \texttt{clustDist}
\begin{verbatim}
make test
\end{verbatim}
\item The executable \texttt{clustDist} is located in the
  directory \texttt{build}. Place it into your \texttt{PATH}.
\item Make the documentation
\begin{verbatim}
make doc
\end{verbatim}
This calls the typesetting program \texttt{latex}, so please make sure
it is installed before making the documentation. The typeset manual is
located in
\begin{verbatim}
doc/clustDist.pdf
\end{verbatim}
\end{itemize}

\section{Tutorial}
\begin{itemize}
  \item Cluster the example distances between 29 \emph{E. coli} and
    \emph{Shigella} genomes.
\begin{verbatim}
clustDist data/eco29.dist
\end{verbatim}
This generates the tree in Newick format:
\begin{verbatim}
(CP000038:0.004996,(CP000036:0.001405,CP001063:0.001640):
0.004064,((CP000800:0.003409,(AP009240:0.003402,CU928160:
0.002841):0.000422):0.002092,((CP000266:0.000595,(AE005674:
0.000242,AE014073:0.000198):0.000501):0.006714,(((CU928163:
0.010177,((FM180568:0.006369,((CU928162:0.004850,(CU928161:
0.000859,(CP000243:0.000558,CP000468:0.000618):0.000416):
0.003651):0.000241,(AE014075:0.004461,CP000247:0.004615):
0.000195):0.001596):0.006307,(CP000970:0.007222,CU928164:
0.006909):0.003175):0.001450):0.003632,(CP000034:0.007614,
(CP001846:0.001439,(AE005174:0.000123,BA000007:0.000425):
0.001037):0.005633):0.002246):0.001902,(((CP000948:0.000021,
CP001396:0.000099):0.000018,(AP009048:0.000008,U00096:
0.0):0.000008):0.004598,(CP000802:0.002850,CP000946:
0.003430):0.001083):0.001768):0.000747):0.000492):0.000335);
\end{verbatim}
\item Draw this tree using
  \texttt{new2view}\footnote{\texttt{https://github.com/evolbioinf/new2view}}
\begin{verbatim}
clustDist data/eco29.dist | new2view
\end{verbatim}
to obtain Figure~\ref{fig:tre}.
\end{itemize}

\begin{figure}
  \begin{center}
    \begin{pspicture}(-4.139,-2.589)(1.530,5.500)
\psline(0.988518,5.200000)(1.529982,5.200000)
\rput(1.259250,5.400000){0.002}
\rput(0.849572,-1.052469){\rnode{n1}{}}\uput{4pt}[-51.088882]{-51.088882}(n1){CP000038}
\rput(1.225235,-0.828351){\rnode{n3}{}}\uput{4pt}[-38.675087]{-38.675087}(n3){CP000036}
\rput(1.326446,-0.787106){\rnode{n5}{}}\uput{4pt}[-26.261293]{-26.261293}(n5){CP001063}
\rput(0.928273,-0.590651){\rnode{n4}{}}
\rput(1.391489,-0.178387){\rnode{n6}{}}\uput{4pt}[-13.847504]{-13.847504}(n6){CP000800}
\rput(1.529982,0.028967){\rnode{n8}{}}\uput{4pt}[-1.433711]{-1.433711}(n8){AP009240}
\rput(1.364309,0.198510){\rnode{n10}{}}\uput{4pt}[10.980083]{10.980083}(n10){CU928160}
\rput(0.609240,0.052012){\rnode{n9}{}}
\rput(0.495388,0.042505){\rnode{n7}{}}
\rput(1.426037,1.229782){\rnode{n12}{}}\uput{4pt}[23.393877]{23.393877}(n12){CP000266}
\rput(1.432101,1.294939){\rnode{n14}{}}\uput{4pt}[35.807671]{35.807671}(n14){AE005674}
\rput(1.414682,1.296582){\rnode{n16}{}}\uput{4pt}[48.221462]{48.221462}(n16){AE014073}
\rput(1.378968,1.256607){\rnode{n15}{}}
\rput(1.278193,1.165823){\rnode{n13}{}}
\rput(0.112263,3.710436){\rnode{n18}{}}\uput{4pt}[60.635254]{60.635254}(n18){CU928163}
\rput(-1.540323,4.890460){\rnode{n20}{}}\uput{4pt}[73.049049]{73.049049}(n20){FM180568}
\rput(-2.147831,5.000000){\rnode{n22}{}}\uput{4pt}[85.462845]{85.462845}(n22){CU928162}
\rput(-2.626342,4.848536){\rnode{n24}{}}\uput{4pt}[87.876640]{277.876648}(n24){CU928161}
\rput(-2.697107,4.860659){\rnode{n26}{}}\uput{4pt}[100.290428]{290.290436}(n26){CP000243}
\rput(-2.735119,4.859754){\rnode{n28}{}}\uput{4pt}[112.704224]{302.704224}(n28){CP000468}
\rput(-2.644720,4.718966){\rnode{n27}{}}
\rput(-2.594472,4.618172){\rnode{n25}{}}
\rput(-2.251701,3.691065){\rnode{n23}{}}
\rput(-3.132794,4.513007){\rnode{n30}{}}\uput{4pt}[125.118011]{315.118011}(n30){AE014075}
\rput(-3.331168,4.331502){\rnode{n32}{}}\uput{4pt}[137.531799]{327.531799}(n32){CP000247}
\rput(-2.277039,3.660770){\rnode{n31}{}}
\rput(-2.235824,3.627780){\rnode{n29}{}}
\rput(-2.043045,3.241081){\rnode{n21}{}}
\rput(-4.122184,2.515731){\rnode{n34}{}}\uput{4pt}[149.945602]{339.945618}(n34){CP000970}
\rput(-4.139389,2.093958){\rnode{n36}{}}\uput{4pt}[162.359390]{352.359375}(n36){CU928164}
\rput(-2.285509,1.845260){\rnode{n35}{}}
\rput(-1.450918,1.639531){\rnode{n33}{}}
\rput(-1.238816,1.309202){\rnode{n19}{}}
\rput(-3.412394,0.016234){\rnode{n38}{}}\uput{4pt}[174.773193]{364.773193}(n38){CP000034}
\rput(-3.056375,-0.680651){\rnode{n40}{}}\uput{4pt}[187.186996]{377.187012}(n40){CP001846}
\rput(-2.940826,-0.746238){\rnode{n42}{}}\uput{4pt}[199.600769]{389.600769}(n42){AE005174}
\rput(-2.997359,-0.806802){\rnode{n44}{}}\uput{4pt}[212.014557]{402.014557}(n44){BA000007}
\rput(-2.911872,-0.729790){\rnode{n43}{}}
\rput(-2.684189,-0.565533){\rnode{n41}{}}
\rput(-1.358191,0.187763){\rnode{n39}{}}
\rput(-0.800112,0.429194){\rnode{n37}{}}
\rput(-0.804597,-1.569357){\rnode{n46}{}}\uput{4pt}[224.428360]{414.428345}(n46){CP000948}
\rput(-0.811830,-1.589375){\rnode{n48}{}}\uput{4pt}[236.842133]{426.842133}(n48){CP001396}
\rput(-0.801290,-1.564733){\rnode{n47}{}}
\rput(-0.799475,-1.564773){\rnode{n50}{}}\uput{4pt}[249.255920]{439.255920}(n50){AP009048}
\rput(-0.799071,-1.562645){\rnode{n52}{}}\uput{4pt}[271.669708]{271.669708}(n52){U00096}
\rput(-0.799071,-1.562645){\rnode{n51}{}}
\rput(-0.798900,-1.560486){\rnode{n49}{}}
\rput(-0.146536,-1.393144){\rnode{n54}{}}\uput{4pt}[284.083527]{284.083527}(n54){CP000802}
\rput(0.080014,-1.475816){\rnode{n56}{}}\uput{4pt}[296.497314]{296.497314}(n56){CP000946}
\rput(-0.334291,-0.644751){\rnode{n55}{}}
\rput(-0.435967,-0.369742){\rnode{n53}{}}
\rput(-0.398103,0.107412){\rnode{n45}{}}
\rput(-0.195930,0.102352){\rnode{n17}{}}
\rput(-0.070806,0.056676){\rnode{n11}{}}
\rput(0.000000,0.000000){\rnode{n2}{}}
\ncline{n1}{n2}
\ncline{n3}{n4}
\ncline{n5}{n4}
\ncline{n4}{n2}
\ncline{n6}{n7}
\ncline{n8}{n9}
\ncline{n10}{n9}
\ncline{n9}{n7}
\ncline{n7}{n11}
\ncline{n12}{n13}
\ncline{n14}{n15}
\ncline{n16}{n15}
\ncline{n15}{n13}
\ncline{n13}{n17}
\ncline{n18}{n19}
\ncline{n20}{n21}
\ncline{n22}{n23}
\ncline{n24}{n25}
\ncline{n26}{n27}
\ncline{n28}{n27}
\ncline{n27}{n25}
\ncline{n25}{n23}
\ncline{n23}{n29}
\ncline{n30}{n31}
\ncline{n32}{n31}
\ncline{n31}{n29}
\ncline{n29}{n21}
\ncline{n21}{n33}
\ncline{n34}{n35}
\ncline{n36}{n35}
\ncline{n35}{n33}
\ncline{n33}{n19}
\ncline{n19}{n37}
\ncline{n38}{n39}
\ncline{n40}{n41}
\ncline{n42}{n43}
\ncline{n44}{n43}
\ncline{n43}{n41}
\ncline{n41}{n39}
\ncline{n39}{n37}
\ncline{n37}{n45}
\ncline{n46}{n47}
\ncline{n48}{n47}
\ncline{n47}{n49}
\ncline{n50}{n51}
\ncline{n52}{n51}
\ncline{n51}{n49}
\ncline{n49}{n53}
\ncline{n54}{n55}
\ncline{n56}{n55}
\ncline{n55}{n53}
\ncline{n53}{n45}
\ncline{n45}{n17}
\ncline{n17}{n11}
\ncline{n11}{n2}
\end{pspicture}

  \end{center}
  \caption{Phylogeny of 29 \textit{E. coli}/\textit{Shigella} strains.}\label{fig:tre}
\end{figure}

\section{Change Log}
Please use
\begin{verbatim}
git log
\end{verbatim}
to list the change history.

\bibliography{ref}
\end{document}

